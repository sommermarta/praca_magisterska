\documentclass[12pt,a4paper]{article}

\usepackage[T1]{fontenc}
\usepackage[polish]{babel}
\usepackage[utf8]{inputenc}
\usepackage{amsmath}
\usepackage{amsfonts}
\usepackage{graphicx}
\usepackage{setspace}
\usepackage{savesym}
%\usepackage{amssymb}
%\savesymbol{arc}
\usepackage{color}
\usepackage{xcolor}
\usepackage{pict2e}
\usepackage{epstopdf}
\usepackage{geometry}
\usepackage{MnSymbol}


\newgeometry{tmargin=1.5cm, bmargin=1.5cm, lmargin=1.5cm, rmargin=1.5cm}
\pagestyle{empty}
\linespread{1.2}
\setlength{\parindent}{0mm}

\newenvironment{dow}{\textbf{\textit{Dowód}}}{\begin{flushright} $\blacksquare$ \end{flushright}}

\begin{document}

% \begin{minipage}[t]{0.5\textwidth}
% \end{minipage}

\section*{\centering MODELOWANIE REGRESJI PORZĄDKOWEJ \\PRZY UŻYCIU PROCESU GAUSSOWSKIEGO}

$\lbrace f(x_i) \rbrace_{i=1}^{n}$ - realizacja procesu gaussowskiego o średniej $0$ i macierzy kowariancji $\Sigma$ zadanej wzorem:

$$
\Sigma = \left( K(x_i, x_j) \right)_{i, j=1\ldots n} = \left(e^{-\frac{\kappa}{2}\sum_{\xi=1}^{d}(x_i^\xi-x_i^\xi)^2}\right)_{i, j=1\ldots n},
$$
gdzie  $\kappa>0$, a $x_i^\xi$ to $\xi$-ty element $x_i$.

Wtedy $f$ ma rozkład łączny o gęstości:

$$
\mathbb{P}(f)=\frac{1}{Z_f}e^{-\frac{1}{2}f^T\Sigma^{-1}f}, 
$$
gdzie $Z_f=(2\Pi)^{\frac{n}{2}}|\Sigma|^\frac{1}{2}$, a $f=[f(x_1),\ldots,f(x_n)]^T$.

Wtedy:
$$
\mathbb{P}(\mathcal{D}|f) = \prod_{i=1}^n\mathbb{P}(y_i|f(x_i)),
$$
gdzie $\mathcal{D} = \lbrace y_1, \ldots, y_r \rbrace$.

Intuicyjnie:

$$
\mathbb{P}_{ideal} (y_i|f(x_i)) = \mathbb{I}\lbrace f(x_i)\in (b_{y_i-1}, b_{y_i}] \rbrace,
$$
gdzie $b_0=-\infty, b_r=+\infty$. 

Można wygodniej sparametryzować $b_i$ jako: $b_1\in\mathbb{R}$, $b_i = \sum_{t=2}^j\Delta_t+b_1$, gdzie $\Delta_t>0$ oraz $j=2, \ldots, r-1$. 

Bardzo rzadko mamy jednak do czynienia z idealną sytuacją, dlatego będziemy budować model zakładając dodatkowy szum $\delta$ o rozkładzie $\mathcal{N}(0, \sigma^2)$. Wtedy prawdopodobieństwo zmienia się następująco:

$$
\mathbb{P} (y_i|f(x_i)) = \Phi\left( z_1^i \right) - \Phi\left( z_2^i \right),
$$
gdzie $z_1^i :=\frac{b_{y_i}-f(x_i)}{\sigma} $ oraz $z_2^i :=\frac{b_{y_i-1}-f(x_i)}{\sigma} $.
\\

\begin{dow}
\begin{align*}
\mathbb{P} (y_i|f(x_i)) 
&= \int \mathbb{P}_{ideal} (y_i |f(x_i)+\delta_i)d\delta_i = \int \mathbb{P}(\delta_i) \mathbb{I}\lbrace f(x_i)+\delta_i\in (b_{y_i-1}, b_{y_i}] \rbrace d\delta_i =\\
&= \int \frac{1}{2\Pi\sigma}e^{-\frac{u^2}{2\sigma^2}} \mathbb{I}\lbrace u\in (b_{y_i-1}-f(x_i), b_{y_i}-f(x_i)] \rbrace du = \int \limits_{b_{y_i}-f(x_i)}^{b_{y_i-1}-f(x_i)}\frac{1}{2\Pi\sigma}e^{-\frac{u^2}{2\sigma^2}} du =\\
&=  \int \limits_{\frac{b_{y_i}-f(x_i)}{\delta}}^{\frac{b_{y_i-1}-f(x_i)}{\delta}}\frac{1}{2\Pi}e^{-\frac{u^2}{2}} du = \Phi\left( \frac{b_{y_i}-f(x_i)}{\sigma} \right) - \Phi\left( \frac{b_{y_i-1}-f(x_i)}{\sigma} \right)
\end{align*}

\end{dow}

Wprowadźmy następującą funkcję straty:
$$
\textit{l}(y_i, f(x_i)) := -\ln\mathbb{P}(y_i| f(x_i))
$$

Jej pochodne to:

\begin{align*}
\frac{\partial\textit{l}(y_i, f(x_i))}{\partial f(x_i)} 
&= \frac{1}{\sigma}\frac{\frac{1}{2\Pi}e^{-\frac{z_1^{i^2}}{2}}-\frac{1}{2\Pi}e^{-\frac{z_2^{i^2}}{2}}}{\Phi\left( z_1^i \right) - \Phi\left( z_2^i \right)}\\
\frac{\partial^2\textit{l}(y_i, f(x_i))}{\partial^2 f(x_i)} 
&= \frac{1}{\sigma^2}\left( \frac{\frac{1}{2\Pi}e^{-\frac{z_1^{i^2}}{2}}-\frac{1}{2\Pi}e^{-\frac{z_2^{i^2}}{2}}}{\Phi\left( z_1^i \right) - \Phi\left( z_2^i \right)} \right)^2 + \frac{1}{\sigma^2}\frac{z_1^i\frac{1}{2\Pi}e^{-\frac{z_1^{i^2}}{2}}-z_2^i\frac{1}{2\Pi}e^{-\frac{z_2^{i^2}}{2}}}{\Phi\left( z_1^i \right) - \Phi\left( z_2^i \right)}
\end{align*}

\begin{dow}
\begin{align*}
\frac{\partial\textit{l}(y_i, f(x_i))}{\partial f(x_i)} 
&= -\ln\left[\Phi\left( z_1^i \right) - \Phi\left( z_2^i \right)\right] = -\frac{1}{\Phi\left( z_1^i \right) - \Phi\left( z_2^i \right)}\cdot \Phi'\left( z_1^i \right)\cdot \left( -\frac{1}{\sigma} \right) - \Phi'\left( z_2^i \right)\cdot \left( -\frac{1}{\sigma} \right) = \\
&= \frac{1}{\sigma}\frac{\frac{1}{2\Pi}e^{-\frac{z_1^{i^2}}{2}}-\frac{1}{2\Pi}e^{-\frac{z_2^{i^2}}{2}}}{\Phi\left( z_1^i \right) - \Phi\left( z_2^i \right)}
\end{align*}

\begin{align*}
\frac{\partial^2\textit{l}(y_i, f(x_i))}{\partial^2 f(x_i)} 
&= \frac{\partial}{\partial f(x_i)}\left( \frac{\partial\textit{l}(y_i, f(x_i))}{\partial f(x_i)} \right) = \frac{\partial}{\partial f(x_i)}\left( \frac{1}{\sigma}\frac{\frac{1}{2\Pi}e^{-\frac{z_1^{i^2}}{2}}-\frac{1}{2\Pi}e^{-\frac{z_2^{i^2}}{2}}}{\Phi\left( z_1^i \right) - \Phi\left( z_2^i \right)} \right) =\\
&= \frac{1}{\sigma}\frac{1}{\left[\Phi\left( z_1^i \right) - \Phi\left( z_2^i \right)\right]^2} \left\lbrace  
\vphantom{\left( 
\frac{1}{2\Pi}
e^{-\frac{z_1^{i^2}}{2}}
-
\frac{1}{2\Pi}
e^{-\frac{z_2^{i^2}}{2}} 
\right)
\cdot
\left(
-\frac{1}{\sigma} 
\right) 
\cdot 
\left( 
\frac{1}{2\Pi}e^{-\frac{z_1^{i^2}}{2}}-\frac{1}{2\Pi}e^{-\frac{z_2^{i^2}}{2}} 
\right)}
  \left[\Phi\left( z_1^i \right) - \Phi\left( z_2^i \right)\right]\cdot\right.\\
&\cdot \left[ 
\dfrac{1}{2\Pi}e^{-\frac{z_1^{i^2}}{2}}\left(-\frac{1}{\not{2}}\cdot \not{2}z_1^i\right)\left(-\frac{1}{\sigma} \right) -     
 \dfrac{1}{2\Pi}e^{-\frac{z_2^{i^2}}{2}}\left(-\frac{1}{\not{2}}\cdot \not{2}z_2^i\right)\left(-\frac{1}{\sigma} \right) \right] -\\
&- 
\left. 
\left( 
\frac{1}{2\Pi}
e^{-\frac{z_1^{i^2}}{2}}
-
\frac{1}{2\Pi}
e^{-\frac{z_2^{i^2}}{2}} 
\right)
\cdot
\left(
-\frac{1}{\sigma} 
\right) 
\cdot 
\left( 
\frac{1}{2\Pi}e^{-\frac{z_1^{i^2}}{2}}-\frac{1}{2\Pi}e^{-\frac{z_2^{i^2}}{2}} 
\right)
\right\rbrace =\\
&=\frac{1}{\sigma^2}\left( \frac{\frac{1}{2\Pi}e^{-\frac{z_1^{i^2}}{2}}-\frac{1}{2\Pi}e^{-\frac{z_2^{i^2}}{2}}}{\Phi\left( z_1^i \right) - \Phi\left( z_2^i \right)} \right)^2 + \frac{1}{\sigma^2}\frac{z_1^i\frac{1}{2\Pi}e^{-\frac{z_1^{i^2}}{2}}-z_2^i\frac{1}{2\Pi}e^{-\frac{z_2^{i^2}}{2}}}{\Phi\left( z_1^i \right) - \Phi\left( z_2^i \right)}
\end{align*}

\end{dow}

Prawdopodobieństwo a posteriori wygląda następująco:
$$
\mathbb{P}(f|\mathcal{D}) = \frac{\mathbb{P}(f)\prod_{i=1}^n\mathbb{P}(y_i|f(x_i))}{\mathbb{P}(\mathcal{D})}
$$

\end{document}
















